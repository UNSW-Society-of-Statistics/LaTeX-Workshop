\documentclass{beamer}

\usepackage{amsmath}
\usepackage{amssymb}
\usepackage{microtype}

\usepackage{parskip}

\usepackage{graphicx}
\usepackage{listings}

\lstset{
	frame=single,
	basicstyle=\tiny\ttfamily,
	tabsize=2
}

\usetheme{Boadilla}
\usecolortheme{spruce}
\usefonttheme[onlymath]{serif}

\author[Nick]{Nicholas Berridge-Argent}
\institute[StatSoc]{UNSW Society of Statistics}
\title{\LaTeX{} Workshop}
\date{2019}

\logo{\includegraphics[width=4cm]{statsoc.png}}

\begin{document}

\AtBeginSection[]{
	\begin{frame}
		\frametitle{Schedule}
		\tableofcontents[currentsection]
	\end{frame}
}

\begin{frame}
	\titlepage
\end{frame}

\begin{frame}
	\begin{center}
		{\Huge{}``This is not a \LaTeX{} course''}

		\hspace*{4cm}-- \textit{every course using \LaTeX}
	\end{center}
\end{frame}

\section{Introduction}

\begin{frame}
	\frametitle{History}
	\begin{columns}
		\begin{column}{0.48\textwidth}
			\begin{footnotesize}
				\begin{itemize}
					\item Donald Knuth was writing the second edition of his book \textit{The Art of Computer Programming} in 1977, when he found that the typesetting used by the publishers was inferior.
					\item As you do, invented a markup language for typesetting called TeX, which was eventually used around the world for typesetting academic documents.
					\item These days, \LaTeX{} forms the most commonly-used extension to plain TeX. Industry-standard for academic documents in mathematics, statistics, computer science, physics, etc.
				\end{itemize}
			\end{footnotesize}
		\end{column}
		\begin{column}{0.48\textwidth}
			\begin{center}
				\includegraphics[width=4cm]{knuth.jpg}
				% image source: wikimedia
			\end{center}
		\end{column}
	\end{columns}
\end{frame}

\begin{frame}
	\frametitle{Environments}
	\begin{itemize}
		\item \textbf{Overleaf} --- fully online, compiles in real time. UNSW Students get free Overleaf Pro account with their UNSW e-mail.
		\item \textbf{TeX Live} --- can be installed on your computer, so no size limits, requires no online access. \textbf{But!} takes a long time to download, takes some skill to set up packages, needs you to compile \LaTeX{} manually. Available for Windows and Linux, Mac users use \textbf{MacTeX}.
		\item \textbf{MiKTeX} --- Alternative to TeX Live for Windows, Linux and Mac.
		\item \textbf{Microsoft Word} --- Not even once... (although the Word equation editor uses TeX-like syntax under the hood)
	\end{itemize}
\end{frame}

\begin{frame}
	\frametitle{Getting Started}
	\begin{columns}
		\begin{column}{0.48\textwidth}
			\begin{center}
				\lstinputlisting{examples/hello.tex}
			\end{center}
		\end{column}
		\begin{column}{0.48\textwidth}
			\begin{center}
				\textrm{Hello, \LaTeX{}!}
			\end{center}
		\end{column}
	\end{columns}
\end{frame}

\begin{frame}
	\frametitle{Getting Started II}
	\begin{columns}
		\begin{column}{0.48\textwidth}
			\begin{center}
				\lstinputlisting{examples/maths.tex}
			\end{center}
		\end{column}
		\begin{column}{0.48\textwidth}
			\begin{center}
				\[
					\sum^n_{k=1}k^2=\frac{n}{6}(2n+1)(n+1)
				\]
			\end{center}
		\end{column}
	\end{columns}
\end{frame}

\section{Document Setup and Text}

\begin{frame}
	\frametitle{The Preamble}
	\begin{itemize}
		\item (Relatively) small section of code for setting the document class, metadata, setting up packages.
		\item \texttt{\textbackslash{}documentclass} --- declares the type of document and sets up a bunch of formatting, most common is \texttt{article}.
		\item \texttt{\textbackslash{}usepackage} --- includes useful packages, most documents have several.
		\item \texttt{\textbackslash{}title}, \texttt{\textbackslash{}author}, \texttt{\textbackslash{}date} --- used to set document metadata.
		\item See example \texttt{theorem.tex}.
	\end{itemize}
\end{frame}

\begin{frame}
	\frametitle{The \texttt{geometry} Package}
	\begin{itemize}
		\item Default paper size in \LaTeX{} is US Letter (215.9 $\times$ 279.4 mm) rather than A4 (210 $\times$ 297 mm).
		\item The margins are set to be enormous based on readability studies, but you may want to change those too.
		\item We can use the package \texttt{geometry} for this:\\\texttt{\textbackslash{}usepackage[a4paper,margin=3cm]\{geometry\}}
		\item Unrelated to the \texttt{geometry} package, but we can also use two columns with:\\\texttt{\textbackslash{}documentclass[twocolumn]\{article\}}
	\end{itemize}
\end{frame}

\begin{frame}
	\frametitle{Metadata}
	\begin{itemize}
		\item Some document metadata can be set in the preamble, then used throughout the document e.g. by the \texttt{\textbackslash{}maketitle} command.
		\item \texttt{\textbackslash{}title} --- Sets document title:\\\texttt{\textbackslash{}title\{I Like Statistics\}}.
		\item \texttt{\textbackslash{}author} --- Sets document author(s):\\\texttt{\textbackslash{}author\{Nicholas Berridge-Argent\textbackslash\textbackslash{}z5208292\}}
		\item \texttt{\textbackslash{}date} --- Sets the date the document was made:\\\texttt{\textbackslash{}date\{\textbackslash{}today\}}
	\end{itemize}
\end{frame}

\begin{frame}
	\frametitle{Essential Packages}
	\begin{itemize}
		\item \texttt{amsmath} and \texttt{amssymb} --- \textbf{very} useful for mathematical typesetting, adds a lot of extra useful environments and symbols.
		\item \texttt{microtype} --- Adjusts fonts to improve readability.
		\item \texttt{parskip} --- Removes indentation at the start of paragraphs (personal preference).
		\item See example \texttt{theorem-improved.tex}.
	\end{itemize}
\end{frame}

\begin{frame}
	\frametitle{Sections and Subsections}
	\begin{itemize}
		\item Sometimes useful to separate a document into several sections.
		\item For this we have \texttt{\textbackslash{}section}, \texttt{\textbackslash{}subsection}, \texttt{\textbackslash{}subsubsection}, \dots
		\item Can be used with an asterisk: \texttt{\textbackslash{}section*} to suppress numbering.
		\item Other document classes provide other forms of sectioning, e.g. \texttt{book} provides \texttt{\textbackslash{}chapter}.
		\item Can be used with \texttt{hyperref} package to add links in a PDF viewer.
	\end{itemize}
\end{frame}

\begin{frame}
	\frametitle{Other Sections}
	\begin{itemize}
		\item We saw before we can add a small abstract with \texttt{abstract}.
		\item We can also define proof / theorem / lemma environments with \texttt{\textbackslash{}newtheorem}
	\end{itemize}
\end{frame}

\begin{frame}
	\frametitle{Lists}
	\begin{itemize}
		\item Un-numbered lists, such as the one you're reading now, can be made with \texttt{itemize}.
		\item Numbered lists can be made with \texttt{enumerate}.
		\item Lists can be nested inside each other.
	\end{itemize}
\end{frame}

\begin{frame}
	\frametitle{Styling Text}
	\begin{itemize}
		\item Text can be made bold or italic with \texttt{\textbackslash{}textbf} and \texttt{\textbackslash{}textit}.
		\item \LaTeX{} supports three default fonts: \textrm{roman}, sans-serif and \texttt{typewriter}.
		\item Change font temporarily with \texttt{\textbackslash{}textrm}, \texttt{\textbackslash{}textsf}, \texttt{\textbackslash{}texttt}.
		\item Change font for whole document in the preamble with \texttt{\textbackslash{}renewcommand\{\textbackslash{}rmdefault\}}, etc.
		\item Can also change font size: \texttt{\textbackslash{}tiny}, \texttt{\textbackslash{}large}, \texttt{\textbackslash{}Huge}, etc.
		\item Can also use \texttt{\textbackslash{}textsuperscript} for superscript, e.g. 1\textsuperscript{st}.
		\item See example \texttt{theorem-final.tex}
	\end{itemize}
\end{frame}

\section{Equations and Maths}

\begin{frame}
	\frametitle{Basic Equations and Symbols}
	\begin{itemize}
		\item Use \texttt{\textbackslash{}(\dots\textbackslash{})} or \texttt{\$\dots\$} for inline equations.
		\item Use \texttt{\textbackslash{}[\dots\textbackslash{}]} or \texttt{\$\$\dots\$\$} for block equations.
		\item Most normal symbols are available just as-is: e.g. \texttt{(3x+2)\^{}2=0} becomes $(3x+2)^2=0$.
		\item Greek letters are available by a backslash followed by their name, e.g. \texttt{\textbackslash{}alpha} becomes $\alpha$.
		\item Fractions are available using \texttt{\textbackslash{}frac}, e.g. \texttt{\textbackslash{}frac\{\textbackslash{}pi\}\{6\}} becomes $\frac{\pi}{6}$.
	\end{itemize}
\end{frame}

\begin{frame}
	\frametitle{Basic Equations and Symbols}
	\begin{itemize}
		\item Sums are available with \texttt{\textbackslash{}sum}, e.g.\\\texttt{\textbackslash{}sum\^{}n\_\{k=1\}3\^{}\{-k\}} becomes $\sum^n_{k=1}3^{-k}$.
		\item Integrals are available with \texttt{\textbackslash{}int}, e.g.\\\texttt{\textbackslash{}int\^{}\textbackslash{}infty\_\{-\textbackslash{}infty\}e\^{}\{-x\^{}2\}\textbackslash{};dx} becomes $\int^\infty_{-\infty}e^{-x^2}\;dx$.
		\item Special mathematical typefaces are available, e.g. \texttt{\textbackslash{}mathcal\{P\}} becomes $\mathcal{P}$ and \texttt{\textbackslash{}mathbb\{R\}} becomes $\mathbb{R}$.
		\item See example \texttt{homework.tex}
	\end{itemize}
\end{frame}

\begin{frame}
	\frametitle{Dilemma!}
		\begin{center}
			\lstinputlisting{examples/sin-beginner.tex}
			\[
				sin(x+(\frac{\pi}{2})^n)
			\]
		\end{center}

		\textbf{Problem:} The sine function is appearing in italics, and the brackets aren't sized properly to the fraction.
\end{frame}

\begin{frame}
	\frametitle{The Solution}
		\begin{center}
			\lstinputlisting{examples/sin-fixed.tex}
			\[
				\sin\left(x+\left(\frac{\pi}{2}\right)^n\right)
			\]
		\end{center}

		\textbf{Solution:} Most functions have a built in command form, and \texttt{\textbackslash{}left} and \texttt{\textbackslash{}right} will automatically size brackets.

		If your function doesn't have a command, you can use \texttt{\textbackslash{}text}, e.g. \texttt{\textbackslash{}text\{cis \}(\textbackslash{}pi/2)} becomes $\text{cis }(\pi/2)$, make sure to include the space inside the \texttt{\textbackslash{}text} command for spacing.
\end{frame}

\begin{frame}
	\frametitle{AMSMath Equation Environments}
	\begin{itemize}
		\item \texttt{equation} provides an enviroment similar to a block equation, but with a unique number on the right-hand margin, useful to refer to multiple equations in text.
		\item \texttt{align} and \texttt{align*} (no numbering) provide block equations which can have multiple lines, aligned at a certain position, useful for showing long streams of working out.
	\end{itemize}
\end{frame}

\begin{frame}
	\frametitle{Environments for Maths Mode}
	\begin{itemize}
		\item \texttt{pmatrix} is useful for setting up vectors and matrices.
		\item \texttt{bmatrix} can also be used for matrices.
		\item \texttt{cases} is useful for piecewise functions.
		\item See example \texttt{homework-2.tex}
	\end{itemize}
\end{frame}

\section{Tables and Images}

\begin{frame}
	\frametitle{Tables and Images}
	\begin{itemize}
		\item Motto of tables in \LaTeX{}: Just use \texttt{booktabs}.
		\item Tables are an environment, \texttt{tabular}, which is set up with specifications for the columns.
		\item Horizontal rules can be used to separate rows.
		\item Vertical rules can be used to separate columns, but please don't.
		\item Different column types available: \texttt{l}, \texttt{c}, \texttt{r}, \texttt{p}.
		\item Images can be inserted using the \texttt{graphicx} package.
	\end{itemize}
\end{frame}

\begin{frame}
	\frametitle{Figures}
	\begin{itemize}
		\item It is often useful to control the positioning of images and tables, and to give them captions.
		\item We can use the \texttt{figure} and \texttt{table} environments for this.
		\item Takes an optional position specifier: \texttt{h}, \texttt{t}, \texttt{b}, \texttt{p}.
		\item Usually useful with the \texttt{\textbackslash{}centering} command or \texttt{center} environment.
		\item Can be combined with a \texttt{\textbackslash{}caption}.
		\item See example \texttt{report.tex}
	\end{itemize}
\end{frame}

\section{Graphs and Diagrams}

\begin{frame}
	\frametitle{Graphs with PGFPlots}
	\begin{itemize}
		\item \LaTeX{} is good for typesetting paragraphs, equations. Not so good for describing graphs.
		\item We need an entirely new set of commands for describing graphs --- PGFPlots.
		\item Can graph raw data (e.g. scatter plots) or mathematical functions.
		\item Can be combined with ``PGF libraries'' to plot graphs more easily.
		\item Can we do 3D graphs too? \textbf{Yes!}
		\item See example \texttt{report-graph.tex}
	\end{itemize}
\end{frame}

\begin{frame}
	\frametitle{Graphs with PGFPlots}
	\begin{itemize}
		\item Inside a \texttt{tikzpicture} environment, have an \texttt{axis} environment.
		\item Use \texttt{\textbackslash{}addplot} to add a plot based on a function, or with \texttt{table} to add a scatterplot.
		\item Use \texttt{\textbackslash{}addlegendentry} to add a reference to that to the legend.
	\end{itemize}
\end{frame}

\begin{frame}
	\frametitle{Graphs with PGFPlots}
	\begin{itemize}
		\item What's the best way to learn the right syntax?
		\item Google.
		\item Doesn't this take a long time?
		\item Yes, feel free to use \texttt{gnuplot}, RStudio, LibreOffice Calc, Microsoft Excel, etc. and \texttt{\textbackslash{}includegraphics}.
		\item But, if you want your documents to look \textbf{really} nice\dots
	\end{itemize}
\end{frame}

\begin{frame}
	\frametitle{Diagrams with TikZ}
	\begin{itemize}
		\item \LaTeX{} is good for typesetting paragraphs, equations. Not so good for describing diagrams.
		\item We need an entirely new set of commands for describing diagrams --- TikZ.
		\item Uses simple drawing commands to draw simple shapes.
		\item Can be combined with ``TikZ packages'' to draw complicated diagrams more easily.
		\item As before, you can use another software package which generates images and \texttt{\textbackslash{}includegraphics}, but you won't get the best looking document possible.
	\end{itemize}
\end{frame}

\begin{frame}
	\frametitle{Diagrams with TikZ}
	\begin{itemize}
		\item Again use a \texttt{tikzpicture} environment.
		\item Use \texttt{\textbackslash{}draw} to draw outlines, \texttt{\textbackslash{}filldraw} to draw filled shapes.
		\item Mostly based on co-ordinates, but can add shapes e.g. circles.
		\item Use \texttt{\textbackslash{}node} to add text.
		\item See example \texttt{report-diagram.tex}
	\end{itemize}
\end{frame}

\section{Extra Document Features}

\begin{frame}
	\frametitle{Table of Contents}
	\begin{itemize}
		\item Very long documents can be hard to nagivate.
		\item Can also be very long to type up a table of contents by hand.
		\item Document is already populated with \texttt{\textbackslash{}section}s, \texttt{\textbackslash{}subsection}s, etc.
		\item We can automatically generate a table of contents based on that.
		\item Sometimes takes 2--3 goes to compile the document.
	\end{itemize}
\end{frame}

\begin{frame}
	\frametitle{Headers and Footers}
	\begin{itemize}
		\item Some assignments specifically require you to add a header and footer.
		\item Other times it can just make your document look very nice.
		\item Using the package \texttt{fancyhdr} you can style custom headers and footers.
	\end{itemize}
\end{frame}

\begin{frame}
	\frametitle{Section Titles}
	\begin{itemize}
		\item The default look of section, subsection headings can be a bit boring.
		\item You might want to use alternative counters e.g. roman numerals.
		\item Using the package \texttt{titlesec} you can style custom section titles.
	\end{itemize}
\end{frame}

\begin{frame}
	\frametitle{Links and References}
	\begin{itemize}
		\item As mentioned before, \texttt{hyperref} can add links to a document in a PDF viewer.
		\item Can also make e.g. a table of contents clickable.
		\item Can \textbf{also} add clickable hyperlinks to web pages.
		\item Quickest way to add references: \texttt{\textbackslash{}footnote}.
		\item Better way to add references: \texttt{thebibliography} environment.
		\item Automatically manages references in the text and adds a bibliography at the end.
	\end{itemize}
\end{frame}

\begin{frame}
	\frametitle{Extra page styling}
	\begin{itemize}
		\item Sometimes a page contains a wide table or graph, and could be made landscape. Enter: \texttt{pdflscape}.
		\item Sometimes you want to add colours to text (even in equations!). Enter: \texttt{xcolor}.
		\item What about extra fonts? We can add some extra fonts as packages, however, they may not support all characters\dots
		\item Using XeTeX (another set of TeX macros similar to \LaTeX{}) we can use \textbf{any} true-type font, but the drawbacks are the same.
		\item In general: just use the default fonts. You can use sans-serif maths in the default font, if you're one of those people.
	\end{itemize}
\end{frame}

\begin{frame}
	\frametitle{Miscellaneous Syntax}
	\begin{itemize}
		\item In the preamble, \texttt{\textbackslash{}renewcommand} will let you define custom commands! \textbf{Very} useful to avoid repeating yourself.
		\item Comments can be added with \texttt{\%} to explain tricky syntax.
	\end{itemize}
\end{frame}

\section{Source Code Listings, Presentations}

\begin{frame}
	\frametitle{Source Code Listings}
	\begin{itemize}
		\item For computer scientists, or people who like code, you may want to include source code into your documents e.g. in an appendix.
		\item Not just as simple as using \texttt{\textbackslash{}texttt}, you want a specific package for that, \texttt{listings}.
		\item Allows for basic syntax highlighting, but not with the typewriter font :( since it doesn't have a bold weight.
		\item For algorithms, ideally use a different package e.g. \texttt{algorithm2e}.
	\end{itemize}
\end{frame}

\begin{frame}
	\frametitle{Presentations}
	\begin{itemize}
		\item The formatting for presentations is \textbf{very} different to the formatting for articles and books.
		\item But, it would be convenient to be able to use the same typesetting e.g. for equations in a presentation.
		\item Academics often need to create a presentation to convey the same ideas they would in an article.
		\item We can use the package \texttt{beamer} for this (although it's really a document class).
		\item See example: this presentation! \texttt{presentation.tex}
	\end{itemize}
\end{frame}

\section{Where To Next?}

\begin{frame}
	\frametitle{Where To Next?}
	\begin{itemize}
		\item Obviously, this workshop was just a start!
		\item Official package documentation on CTAN: \texttt{https://www.ctan.org/pkg}.
		\item More tutorials on Overleaf website: \texttt{https://www.overleaf.com/learn} and Wikibooks: \texttt{https://en.wikibooks.org/wiki/LaTeX}.
		\item \LaTeX{} Stack Exchange for questions \& answers: \texttt{https://tex.stackexchange.com/}.
		\item OEIS has a list of symbols: \texttt{https://oeis.org/wiki/List\_of\_LaTeX\_mathematical\_symbols}.
		\item Slides and examples will be made available on the StatSoc website: \texttt{https://statsoc.unsw.edu.au/}.
	\end{itemize}
\end{frame}

\end{document}
