\documentclass{article}

\usepackage{amsmath}
\usepackage{amssymb}

\usepackage{microtype}
\usepackage{parskip}

\newtheorem{theorem}{Theorem}[section]

\title{My Theorem: Improved}
\author{Nick}
\date{\today}

\begin{document}

\maketitle

\begin{abstract}
    People have long been searching, everywhere except in a high school maths textbook, for the sum of the first $n$ square numbers. Today, we present a closed formula for this sum.
\end{abstract}

\section{The Theorem}

Today, I am going to present the following theorem for all $n\in\mathbb{N}$:

\[
    \sum^n_{k=1}k^2=\frac{n}{6}(2n+1)(n+1)
\]

This is useful because
\begin{enumerate}
    \item I like squares.
    \item I don't like long sums.
\end{enumerate}

\section{The Proof}

\subsection{Preface}

\textit{Lemma 2.1.1} First, we make use of the convenient lemma $k=k$.

\begin{theorem}[Nick's Theorem]
    Nick's theorem states that $1=1$. Note that this is actually a special case of \textit{Lemma 2.1.1}
\end{theorem}

\subsection{Final Proof}

\textit{Proof.} The proof is left as an exercise for a reader.\hfill$\blacksquare$

\end{document}